% Options for packages loaded elsewhere
\PassOptionsToPackage{unicode}{hyperref}
\PassOptionsToPackage{hyphens}{url}
%
\documentclass[
]{article}
\usepackage{lmodern}
\usepackage{amssymb,amsmath}
\usepackage{ifxetex,ifluatex}
\ifnum 0\ifxetex 1\fi\ifluatex 1\fi=0 % if pdftex
  \usepackage[T1]{fontenc}
  \usepackage[utf8]{inputenc}
  \usepackage{textcomp} % provide euro and other symbols
\else % if luatex or xetex
  \usepackage{unicode-math}
  \defaultfontfeatures{Scale=MatchLowercase}
  \defaultfontfeatures[\rmfamily]{Ligatures=TeX,Scale=1}
\fi
% Use upquote if available, for straight quotes in verbatim environments
\IfFileExists{upquote.sty}{\usepackage{upquote}}{}
\IfFileExists{microtype.sty}{% use microtype if available
  \usepackage[]{microtype}
  \UseMicrotypeSet[protrusion]{basicmath} % disable protrusion for tt fonts
}{}
\makeatletter
\@ifundefined{KOMAClassName}{% if non-KOMA class
  \IfFileExists{parskip.sty}{%
    \usepackage{parskip}
  }{% else
    \setlength{\parindent}{0pt}
    \setlength{\parskip}{6pt plus 2pt minus 1pt}}
}{% if KOMA class
  \KOMAoptions{parskip=half}}
\makeatother
\usepackage{xcolor}
\IfFileExists{xurl.sty}{\usepackage{xurl}}{} % add URL line breaks if available
\IfFileExists{bookmark.sty}{\usepackage{bookmark}}{\usepackage{hyperref}}
\hypersetup{
  pdftitle={Data605-Week5-HomeWork5-kamath},
  pdfauthor={Vinayak Kamath},
  hidelinks,
  pdfcreator={LaTeX via pandoc}}
\urlstyle{same} % disable monospaced font for URLs
\usepackage[margin=1in]{geometry}
\usepackage{graphicx,grffile}
\makeatletter
\def\maxwidth{\ifdim\Gin@nat@width>\linewidth\linewidth\else\Gin@nat@width\fi}
\def\maxheight{\ifdim\Gin@nat@height>\textheight\textheight\else\Gin@nat@height\fi}
\makeatother
% Scale images if necessary, so that they will not overflow the page
% margins by default, and it is still possible to overwrite the defaults
% using explicit options in \includegraphics[width, height, ...]{}
\setkeys{Gin}{width=\maxwidth,height=\maxheight,keepaspectratio}
% Set default figure placement to htbp
\makeatletter
\def\fps@figure{htbp}
\makeatother
\setlength{\emergencystretch}{3em} % prevent overfull lines
\providecommand{\tightlist}{%
  \setlength{\itemsep}{0pt}\setlength{\parskip}{0pt}}
\setcounter{secnumdepth}{-\maxdimen} % remove section numbering

\title{Data605-Week5-HomeWork5-kamath}
\author{Vinayak Kamath}
\date{09/26/2020}

\begin{document}
\maketitle

\hypertarget{problem-set-1}{%
\subsection{Problem set 1}\label{problem-set-1}}

Choose independently two numbers B and C at random from the interval
{[}0, 1{]} with uniform density. Prove that B and C are proper
probability distributions.

Note that the point (B,C) is then chosen at random in the unit square.

Find the probability that

\begin{enumerate}
\def\labelenumi{(\alph{enumi})}
\item
  B + C \textless{} 1/2

  B + C = 1 passes through the points (1/2, 0) and (0, 1/2).

  ==\textgreater{} B + C \textless{} 1 is the area of the triangle
  formed in the unit square and the half plane B + C \textless{} 0.5

  ==\textgreater{} So, P(B + C) = (Area of triangle with vertices (1/2,
  0), (0, 1/2), (0, 0))

  ==\textgreater{} (1/2) * (1/2)(1/2) = \textbf{\emph{1/8}}
\end{enumerate}

\begin{center}\rule{0.5\linewidth}{0.5pt}\end{center}

\begin{enumerate}
\def\labelenumi{(\alph{enumi})}
\setcounter{enumi}{1}
\item
  BC \textless{} 1/2

  ==\textgreater{} P(BC \textless{} 1/2)

  ==\textgreater{} ∫(x = 1/2 to 1) dx/(2x) = \textbf{\emph{(1/2) ln 2}}
\end{enumerate}

\begin{center}\rule{0.5\linewidth}{0.5pt}\end{center}

\begin{enumerate}
\def\labelenumi{(\alph{enumi})}
\setcounter{enumi}{2}
\item
  \textbar B − C\textbar{} \textless{} 1/2

  ==\textgreater{} P(B \textgreater= 1/2 + C, C \textless{} 1/2 ) + P(C
  \textgreater= 1/2 + B, B \textless{} 1/2 )

  ==\textgreater{} P(B \textgreater{} 1/2 , C \textless{} 1/2)
  \textbf{+} P(C \textgreater{} 1/2, B \textless{} 1/2)

  ==\textgreater{} B \textless{} 1/2 or B \textgreater{} 1/2 occurs half
  the time, so the probability is 1/2

  ==\textgreater{} C \textless{} 1/2 or C \textgreater{} 1/2 occurs half
  the time, so the probability is 1/2

  ==\textgreater{} 1/2 \textbf{+} 1/2 = \textbf{\emph{1/4}}
\end{enumerate}

\begin{center}\rule{0.5\linewidth}{0.5pt}\end{center}

\begin{enumerate}
\def\labelenumi{(\alph{enumi})}
\setcounter{enumi}{3}
\item
  max\{B,C\} \textless{} 1/2

  ==\textgreater{} Means P(B \textless{} 1/2) \textbf{and} P(C
  \textless{} 1/2)

  ==\textgreater{} B \textless{} 1/2 occurs half the time, so the
  probability is 1/2

  ==\textgreater{} C \textless{} 1/2 occurs half the time, so the
  probability is 1/2

  ==\textgreater{} 1/2 x 1/2 = \textbf{\emph{1/4}}
\end{enumerate}

\begin{center}\rule{0.5\linewidth}{0.5pt}\end{center}

\begin{enumerate}
\def\labelenumi{(\alph{enumi})}
\setcounter{enumi}{4}
\item
  min\{B,C\} \textless{} 1/2

  ==\textgreater{} Means P(B \textless{} 1/2) \textbf{or} P(C
  \textless{} 1/2)

  ==\textgreater{} B \textless{} 1/2 occurs half the time, so the
  probability is 1/2

  ==\textgreater{} C \textless{} 1/2 occurs half the time, so the
  probability is 1/2

  ==\textgreater{} 1/2 or 1/2 = \textbf{\emph{1/2}}
\end{enumerate}

\begin{center}\rule{0.5\linewidth}{0.5pt}\end{center}

\end{document}
